\documentclass[10pt,twocolumn,letterpaper]{article}
\usepackage[french]{babel}
\usepackage[utf8]{inputenc}
\usepackage{cvpr}
\usepackage{times}
\usepackage{epsfig}
\usepackage{graphicx}
\usepackage{amsmath}
\usepackage{amssymb}

% Include other packages here, before hyperref.

% If you comment hyperref and then uncomment it, you should delete
% egpaper.aux before re-running latex.  (Or just hit 'q' on the first latex
% run, let it finish, and you should be clear).
\usepackage[breaklinks=true,bookmarks=false]{hyperref}

\cvprfinalcopy % *** Uncomment this line for the final submission

\def\cvprPaperID{****} % *** Enter the CVPR Paper ID here
\def\httilde{\mbox{\tt\raisebox{-.5ex}{\symbol{126}}}}

% Pages are numbered in submission mode, and unnumbered in camera-ready
%\ifcvprfinal\pagestyle{empty}\fi
\setcounter{page}{4321}
\begin{document}

\title{Compression d'image avec perte à base d'autoencodeurs}

\author{Romain Meynard\\
{\tt\small romain.meynard@student.ecp.fr}
\and
Wenceslas des Déserts \\
{\tt\small venceslas.danguy-des-deserts@student.ecp.fr}
\and
Silvestre Perret \\
{\tt\small silvestre.perret@student.ecp.fr}	 \\
\\
CentraleSupélec\\
}

\maketitle
%\thispagestyle{empty}

\begin{abstract}
La compression d'image est un sujet de recherche d'actualité dans la mesure où les résolutions d'images augmentent et les débits d'échange également. L'approche de compression avec perte consiste à s'autoriser à ne garder qu'une partie de l'information contenue dans l'image en contrepartie d'un taux de compression bien supérieur. Des algorithmes tels que le JPEG mettent déjà en œuvre cette approche avec des taux particulièrement intéressants. Cependant de récents travaux mettent en lumière les apports potentiels de l'apprentissage profonds à la compression d'images. En effet la compression d'images comme un problème d'optimisation d'un réseau profond permet de tenir compte de nouveaux aspects tels que le contenu sémantique de l'image. Notre travail vise à étudier l'effet de différentes fonctions de coût sur la performance de la compression de l'autoencodeur.

\end{abstract}

\section{Introduction}

\section{Etat de l'art}
L'état de l'art de la compression d'image à base de réseaux profonds reposent principalement sur des structures d'autoencodeurs et de réseaux adversariaux génératifs (GANs). En effet, la structure de ces architectures en deux blocs opposés se prête particulièrement bien à la compression et décompression de l'image. Le premier bloc aura pour charge de produire une version compressée de l'image que le deuxième bloc devra décompresser. En particulier Theis et al. ont développé un autoencodeur qui produit des résultats sur la compression comparable au JPEG 2000. Le problème de la compression d'image est également très proche du problème de super résolution qui consiste à augmenter la résolution d'une image. Dans ce domaine Sajjadi et al. ont produit EnhanceNet en (année), il s'agit d'un GAN qui génère une version agrandie quatre fois d'une image initiale. Ce modèle repose en particulier sur la génération automatique de texture par le réseau. Notre intuition est que certaines approches de résolution du problème de super résolution peuvent être transposées à la compression d'image en particulier la génération de texture.

\section{Modèle}
Notre travail repose principalement sur le papier produit par Theis et al.. Nous souhaitons mesurer les effets de l'ajout de fonction de coût perceptuelle et de texture au modèle initial. En effet ces fonctions sont utilisés dans l'état de l'art et Theis et al. suggère ces pistes d'amélioration dans son papier.

\subsection{Architecture}
 
\subsection{Compression}
hufffman
\subsection{Fonctions de coût}



\section{Expériences}

Which dataset did we use, why and how ?

\section{Résultats}

Probably a bunch of nice images showing how great our autoencoder performs.

\section{Discussion} % Conclusion ?

What did we do ? Are our results good ? What could we do next if we were not writing this report less than four days before the deadline ?

{\small
\bibliographystyle{ieee}
\bibliography{semantic_ae}
}

\end{document}
